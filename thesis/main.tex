\documentclass[a4paper,12pt]{report}

% Подключаем пакет конфигурации из шаблона
\usepackage[russian]{config}
\usepackage{hyperref}

% --- Основные данные ---
\newcommand{\uniGroupName}{I-2402}
\newcommand{\authorName}{Bugaenco Nichita}
\newcommand{\thesisTitle}{Raport de practică: Explorarea formatului \LaTeX{} și a sistemului de control al versiunilor Git}
\newcommand{\thesisType}{practică}
\newcommand{\programulDeStudii}{licență}
\newcommand{\identificatorulCursului}{0613.3 Informatică}
\newcommand{\conducatorName}{Curmanschii Anton}

\newcommand{\authorNameRu}{Бугаенко Никита Игоревич}
\newcommand{\thesisTitleRu}{Отчет по практике: Изучение \LaTeX{} и системы контроля версий Git}
\newcommand{\thesisTypeRu}{практической работе}
\newcommand{\programulDeStudiiRu}{бакалавриата}
\newcommand{\identificatorulCursuluiRu}{0613.4 Информатика}
\newcommand{\conducatorNameRu}{Курманский Антон}

\newcommand{\conferencesList}{Студенческая конференция UTM, \year~года}
\renewcommand{\year}{2025}
\newcommand{\github}{\url{https://github.com/hoholms/uni_thesisTemplate}}
\newcommand{\outputDate}{\today}

\begin{document}

% Генерация титульного листа
    \titlePage

    \clearpage
    \tableofcontents

    \clearpage
    \unnumberedChapter{Аннотация}

    \textbf{к \thesisTypeRu{} ``\thesisTitleRu{}'', студента \authorNameRu{}, группа \uniGroupName{}, программа обучения \programulDeStudiiRu.}

    \textbf{Структура работы.}
    Работа состоит из: Введения, \chapterCount{} глав, Общих выводов и рекомендаций, Библиографии из \bibliographyEntryCount{} наименований и \anexeCount{} приложений.
    Основной текст занимает \usefulPageCount{} страниц.

    \textbf{Ключевые слова:}
    \textit{\LaTeX{}, Git, \ac{VCS}, \ac{PR}, Branch, Commit, Merge, Rebase, Cherry-Pick, Reset, Revert}

    \textbf{Актуальность.}
    В современной разработке программного обеспечения профессиональное владение инструментами для документирования и контроля версий является обязательным. \LaTeX{} представляет собой стандарт де-факто для написания научных и технических работ,
    обеспечивая высокое качество типографики.
    Git является самой популярной системой контроля версий, необходимой для эффективной командной и индивидуальной разработки.
    Освоение этих инструментов на начальном этапе карьеры создает прочный фундамент для будущего профессионального роста.

    \textbf{Цель и задачи.}
    Целью данной работы является документирование процесса изучения и практического применения системы верстки \LaTeX{} и системы контроля версий Git.
    Задачи:
    \begin{itemize}
        \item Изучить и настроить шаблон \LaTeX{} для создания технических отчетов.
        \item Пройти интерактивный курс по ветвлению в Git (Learn Git Branching) и задокументировать ключевые этапы.
        \item Пройти практический курс Git Immersion для закрепления навыков работы с Git и задокументировать выполнение уроков.
    \end{itemize}

    \textbf{Полученные результаты.}
    В ходе выполнения практической работы были получены и закреплены навыки работы с \LaTeX{} для создания структурированных документов и с Git для управления версиями исходного кода.
    Был создан данный отчет, демонстрирующий полученные знания.

    \textbf{Практическая ценность.}
    Результатом работы является структурированный отчет, который может служить методическим пособием для других студентов, начинающих изучение \LaTeX{} и Git.
    Весь исходный код и текст отчета доступны на GitHub, что позволяет использовать его как реальный пример проекта.

    Весь исходный код проекта доступен на GitHub по следующей ссылке: \github.

    \clearpage
    \unnumberedChapter{Список сокращений}
    \begin{acronym}[JPEG]
        \acro{VCS}{Version Control System}
        \acro{PR}{Pull Request}
        \acro{GUI}{Graphical User Interface}
    \end{acronym}

    \introChapter

    \textbf{Актуальность и важность темы.}
    В современном мире IT специалист должен не только писать код, но и уметь грамотно оформлять документацию, а также эффективно управлять историей изменений своего проекта.
    \LaTeX{} позволяет создавать профессионально выглядящие документы, что особенно важно в академической и научной среде.
    Git является индустриальным стандартом для контроля версий, без которого немыслима ни одна серьезная разработка.
    Изучение этих технологий является ключевой инвестицией в профессиональное развитие.

    \textbf{Цель и задачи.}
    Цель: Получить и закрепить практические навыки использования \LaTeX{} и Git.
    Задачи:
    \begin{itemize}
        \item Детально изучить структуру и возможности шаблона \LaTeX{} для дипломных работ.
        \item Освоить механики ветвления в Git через интерактивный тренажер.
        \item Пройти пошаговое руководство по основным командам Git для ежедневного использования.
        \item Систематизировать и задокументировать полученные знания в виде отчета по практике.
    \end{itemize}

    \textbf{Методологическая и технологическая база.}
    \begin{itemize}
        \item Система верстки: \LaTeX{} с дистрибутивом MaKTeX\@.
        \item Редактор кода: IntelliJ IDEA с плагином TeXiFy-IDEA\@.
        \item Шаблон для отчета: \texttt{uni\_thesisTemplate} от Антона Курманского.
        \item Интерактивный тренажер по Git: Learn Git Branching\cite{gitbranching}.
        \item Практическое руководство по Git: Git Immersion\cite{gitimmersion}.
    \end{itemize}

    \textbf{Краткое содержание работы.}

    Первая глава, \nameref{ch:latex_chapter_title}, посвящена изучению шаблона \LaTeX{}.
    В ней описывается процесс установки, настройки и использования основных элементов шаблона: работа с текстом, кодом, изображениями и таблицами.

    Вторая глава, \nameref{ch:git_branching_chapter_title}, документирует прохождение интерактивного курса Learn Git Branching.
    Представлены решения ключевых задач по созданию веток, слиянию, перемещению и другим операциям в Git.

    Третья глава, \nameref{ch:git_immersion_chapter_title}, описывает выполнение практических уроков из руководства Git Immersion, охватывающих основной рабочий процесс с Git, от создания репозитория до работы с удаленными серверами.


% --- ГЛАВА 1: Изучение LaTeX ---


    \chapter{Изучение и настройка шаблона \LaTeX{}}\label{ch:latex_chapter_title}

    \section{Установка и первая компиляция}\label{sec:install-and-first-compile}
Здесь начинается мое знакомство с \LaTeX{}.
Для работы я установил дистрибутив MacTeX и установил плагин TeXiFy-IDEA в IDE IntelliJ IDEA\@.
Процесс установки прошел гладко с использованием утилиты \texttt{brew}.
После установки я склонировал репозиторий с шаблоном и открыл его в редакторе.

На \refFigure{first_compile} показан результат первой успешной компиляции проекта.

\imageWithCaption{latex-compile.png}{Результат первой компиляции}
\label{fig:first_compile}


\section{Структура и кастомизация шаблона}\label{sec:structure-and-customization}
Основной файл \texttt{main.tex} подключает все остальные части.
Конфигурация вынесена в \texttt{config.sty}.
Исходники для примеров кода находятся в папке \texttt{src/}, а изображения в папке \texttt{thesis/images}.

Для изменения личных данных, названия работы и другой информации достаточно отредактировать блок \texttt{\textbackslash newcommand} в преамбуле \texttt{main.tex}.
\begin{minted}{latex}
        % Пример отредактированного блока
        \newcommand{\authorNameRu}{Бугаенко Никита Игоревич}
        \newcommand{\thesisTitleRu}{Отчет по практике: Изучение \LaTeX{} и системы контроля версий Git}
\end{minted}


\section{Документирование: работа с текстом, кодом и изображениями}\label{sec:documenting-with-text-code-images}
В ходе работы над отчетом я ознакомился с основными командами для форматирования текста, создания списков, вставки таблиц, изображений и блоков кода.

\subsection{Вставка блоков кода с помощью \texttt{minted}}\label{subsec:minted-code-blocks}
Пакет \texttt{minted} оказался очень удобным для отображения исходного кода.
Он автоматически подсвечивает синтаксис для множества языков.
\begin{minted}[frame=lines,
    framesep=2mm,
    linenos,
    fontsize=\small]{javascript}
        // Пример кода
        function helloGit() {
            console.log("Hello, Git Immersion!");
        }
\end{minted}

\subsection{Вставка изображений}\label{subsec:inserting-images}
Для вставки изображений (скриншотов) используется команда \texttt{\textbackslash imageWithCaption}.
Она автоматически создает рисунок с подписью и добавляет его в список иллюстраций.
На \refFigure{git_branching_example} приведен пример.

\imageWithCaption{git-branching/basics/c5t3.png}{Пример выполнения задания в Learn Git Branching}
\label{fig:git_branching_example}


\chapterConclusionSection{latex_chapter_title}
В этой главе я рассмотрел процесс установки и настройки рабочего окружения для \LaTeX{}, изучил структуру предоставленного шаблона и освоил базовые команды для наполнения документа контентом: текстом, кодом и изображениями.
Этот опыт стал основой для документирования следующих этапов практики.



% --- ГЛАВА 2: Learn Git Branching ---


    \chapter{Практика с Learn Git Branching}\label{ch:git_branching_chapter_title}

    \section{Введение в Learn Git Branching}\label{sec:--learn-git-branching}
Learn Git Branching — это интерактивный веб-тренажер для изучения Git.
Он визуализирует структуру коммитов и веток, что сильно упрощает понимание команд.
На \refFigure{git-branching/basics/c1t1.png} представлен интерфейс тренажера в начале первого задания.

\imageWithCaption{git-branching/basics/c1t1.png}{Главный экран Learn Git Branching}
\label{fig:lgb_main_screen}


\section{Основные уровни: ветвление и слияние}\label{sec:-:---}
Начальные уровни посвящены базовым командам, таким как создание коммитов, ветвление и слияние.

\subsection{Команды \texttt{git commit} и \texttt{git branch}}\label{subsec:-texttt{git-commit}--texttt{git-branch}}
Первые уроки знакомят с основами: созданием коммитов и веток.
На \refFigure{git-branching/basics/c1t1.png} показано решение задачи на создание коммита.
На \refFigure{git-branching/basics/c1t3.png} — создание новой ветки.

\imageWithCaption{git-branching/basics/c1t1.png}{Создание коммита}
\label{fig:lgb_commit}

\imageWithCaption{git-branching/basics/c1t3.png}{Создание новой ветки}
\label{fig:lgb_branch}

\subsection{Команды \texttt{git merge} и \texttt{git rebase}}\label{subsec:-texttt{git-merge}--texttt{git-rebase}}
Я изучил два способа объединения веток: слияние (\texttt{merge}) и перебазирование (\texttt{rebase}).
\texttt{Merge} создает новый коммит слияния, сохраняя обе истории веток.
\texttt{Rebase} переписывает историю, делая ее линейной.
На \refFigure{git-branching/remotes/c2t2.png} показан результат выполнения команды \texttt{git merge}.
На \refFigure{git-branching/remotes/c2t1.png} — результат \texttt{git rebase}.

\imageWithCaption{git-branching/remotes/c2t2.png}{Результат слияния веток}
\label{fig:lgb_merge}

\imageWithCaption{git-branching/remotes/c2t1.png}{Результат перебазирования ветки}
\label{fig:lgb_rebase}


\section{Продвинутые уровни: перемещение по истории}\label{sec:-:---2}
В продвинутых уроках я познакомился с более сложными техниками, такими как выборочный перенос коммитов и отмена изменений.

\subsection{Команда \texttt{git cherry-pick}}\label{subsec:-texttt{git-cherry-pick}}
Команда \texttt{git cherry-pick} позволяет скопировать коммит из одной ветки в другую.
Это полезно, когда нужно применить конкретное изменение, не сливая всю ветку.
На \refFigure{git-branching/basics/c5t3.png} показано применение этой команды.

\imageWithCaption{git-branching/basics/c5t3.png}{Использование cherry-pick для копирования коммита}
\label{fig:lgb_cherry_pick}

\subsection{Команда \texttt{git reset} и \texttt{git revert}}\label{subsec:-texttt{git-reset}--texttt{git-revert}}
Я также изучил способы отмены изменений. \texttt{git reset} откатывает ветку к определенному коммиту, удаляя последующие. \texttt{git revert} создает новый коммит, который отменяет изменения указанного коммита, сохраняя историю.
На \refFigure{git-branching/basics/c2t4.png} показан пример использования \texttt{git reset} для отката к предыдущему коммиту, а также \texttt{revert} для отмены коммита, создавая новый коммит с противоположными изменениями.

\imageWithCaption{git-branching/basics/c2t4.png}{Использование reset для отката к предыдущему коммиту и revert для отмены коммита (создание нового коммита)}
\label{fig:lgb_reset_revert}


\section{Работа с удаленными репозиториями}\label{sec:---}
Отдельный блок заданий посвящен работе с удаленными репозиториями.
Я освоил команды для клонирования репозитория (\texttt{git clone}), отправки изменений на сервер (\texttt{git push}) и получения обновлений (\texttt{git pull} и \texttt{git fetch}).
На \refFigure{git-branching/remotes/c1t1.png} показан пример клонирования удаленного репозитория.

\imageWithCaption{git-branching/remotes/c1t1.png}{Клонирование удаленного репозитория}
\label{fig:lgb_clone}

На \refFigure{git-branching/remotes/c2t5.png} показан результат отправки своих изменений на удаленный сервер.

\imageWithCaption{git-branching/remotes/c2t5.png}{Отправка изменений с помощью \texttt{git push}}
\label{fig:lgb_push}


\chapterConclusionSection{git_branching_chapter_title}
В этой главе я задокументировал прохождение интерактивного курса Learn Git Branching.
Визуализация процесса помогла мне глубоко понять механику работы веток в Git.
Я освоил как базовые операции (\texttt{commit}, \texttt{branch}, \texttt{merge}), так и более сложные (\texttt{rebase}, \texttt{cherry-pick}).



% --- ГЛАВА 3: Git Immersion ---


    \chapter{Освоение Git Immersion}\label{ch:git_immersion_chapter_title}

    \section{Уроки 1--15: Основы локального репозитория}\label{sec:git-immersion-local-repo}
В данном разделе рассмотрены базовые операции с локальным репозиторием:

\begin{itemize}
    \item \textbf{Setup and More Setup} — установка Git, конфигурация имени и e‑mail.
    \item \textbf{Create a Project} — инициализация нового проекта (\texttt{git init}).
    \item \textbf{Checking Status} — проверка состояния репозитория (\texttt{git status}).
    \item \textbf{Making Changes} — изменение файлов.
    \item \textbf{Staging Changes} — добавление изменений в индекс (\texttt{git add}).
    \item \textbf{Staging and Committing} — подготовка и создание коммитов.
    \item \textbf{Committing Changes} — подробности создания коммита.
    \item \textbf{Changes, not Files} — отслеживание изменений, а не файлов.
    \item \textbf{History} — просмотр истории (\texttt{git log}).
    \item \textbf{Aliases} — настройка («alias») для удобства.
    \item \textbf{Getting Old Versions} — возврат к старым версиям.
    \item \textbf{Tagging versions} — создание тегов для версий.
    \item \textbf{Undoing Local Changes (before staging)} — откат изменений до стадии индексации.
    \item \textbf{Undoing Staged Changes (before committing)} — снятие из индекса.
    \item \textbf{Undoing Committed Changes} — отмена последнего коммита.
\end{itemize}

\paragraph{Описания.} Ниже — демонстрация команд и мест для скриншотов:

\begin{minted}{shell}
# Setup
git config --global user.name "Ваше Имя"
git config --global user.email "email@example.com"

# Initialize
git init myproject

# Check status
cd myproject
git status
# … далее git add, git commit …
\end{minted}

\imageWithCaption{git-immersion/staging.png}{Проверка статуса после инициализации и первые коммиты}
\label{fig:git_immersion_staging}

\paragraph{Настройка алиасов.} В этом разделе я также настроил алиасы для удобства работы с Git.
Ниже приведен пример конфигурации алиасов, которые я использую:
\begin{minted}{shell}
# //=============\\
# || Git Aliases ||
# \\=============//

alias g='git'
alias ga='git add'
alias gc='git commit -v'
alias gcam='git commit -am'
alias gd='git diff'
alias gds='git diff --staged'
alias gf='git fetch'
alias gfp='git fetch && git pull'
alias glgg='git log --graph'
alias glgga='git log --graph --decorate --all'
alias glgm='git log --graph --max-count=10'
alias gp='git push'
alias gpom='git push origin master'
alias grmc='git rm --cached'
alias gst='git status'

\end{minted}

\clearpage


\section{Уроки 16--30: Ветвление, слияние и история}\label{sec:git-immersion-branching-merging}
Этот блок посвящен веткам и управлению ими:

\begin{itemize}
    \item \textbf{Removing Commits from a Branch} — удаление последних коммитов (пример: \texttt{git reset --hard HEAD~2}).
    \item \textbf{Remove the oops tag} — удаление тега (\texttt{git tag -d oops}).
    \item \textbf{Amending Commits} — правка последнего коммита (\texttt{git commit --amend}).
    \item \textbf{Moving Files} — перемещение/переименование в Git (\texttt{git mv}).
    \item \textbf{More Structure} — организация структуры проекта.
    \item \textbf{Creating a Branch} — создание ветки (\texttt{git branch feature1}).
    \item \textbf{Navigating Branches} — переключение между ветками (\texttt{git checkout feature1}).
    \item \textbf{Changes in Main} — внесение изменений в \texttt{main}.
    \item \textbf{Viewing Diverging Branches} — обзор разветвлений (\texttt{git log --graph --decorate --all}).
    \item \textbf{Merging} — слияние веток (\texttt{git merge feature1}).
    \item \textbf{Creating a Conflict} — искусственное создание конфликта (изменение одного и того же файла).
    \item \textbf{Resolving Conflicts} — разрешение конфликта в редакторе, \texttt{git add`, `git commit}.
\end{itemize}

\paragraph{Пример команд:}
\begin{minted}{shell}
git checkout -b feature1
# вносим изменения в файл.txt
git add файл.txt
git commit -m "Добавили в feature1"

git checkout main
echo "Конфликтная строка" >> файл.txt
git commit -am "Изменение в main"

git merge feature1
# появится конфликт…
# → разрешаем, фиксируем, снова commit
\end{minted}

\clearpage


\section{Уроки 31--53: Удалённые репозитории (опционально)}\label{sec:31-53}
Если пройдены уроки с 31–53, опишите работу с GitHub:

\begin{itemize}
    \item git remote add
    \item git push
    \item git pull
    \item git clone
\end{itemize}

\paragraph{Пример:}
\begin{minted}{shell}
# Клонирование
git clone https://github.com/hoholms/uni_thesisTemplate
cd uni_thesisTemplate

# Добавить удалённый
git remote add origin git@github.com:hoholms/uni_thesisTemplate.git

# Отправить изменения
git push -u origin main

# Забрать обновления
git pull origin main
\end{minted}

\clearpage

\section*{Заключение\label{sec:conclusion}}
\addcontentsline{toc}{section}{Заключение}
В этой главе задокументирован процесс прохождения курса Git Immersion: от базовых команд до работы с удалёнными репозиториями.
Это крепко закрепило навыки работы с Git как с \ac{VCS} и подготовило к командной работе в реальном проекте.

В качестве доказательства навыков работы с Git (помимо задач выполненных во время практики), представляю ссылку на репозиторий с исходным кодом и текстом отчета: \github.
Также, в профиле GitHub можно найти другие проекты, где я применяю Git для контроля версий.

\section*{Сертификат}

На \refFigure{certif-git.jpg} представлен сертификат об успешном прохождении курса по Git на Udemy\cite{udemygitbootcamp} (\href{https://www.udemy.com/certificate/UC-43d72581-dc6c-4add-90a7-48f60475e13f/}{Ссылка на сертификат об окончании курса Udemy}):

\imageWithCaption{certif-git.jpg}{\href{https://www.udemy.com/course/git-and-github-bootcamp/}{The Git \& Github Bootcamp} Udemy сертификат}
\label{fig:udemy_certif_git}



% --- Заключение ---
    \unnumberedChapter{Заключительные выводы и рекомендации}

    В ходе выполнения данной практической работы я достиг поставленных целей: изучил основы системы верстки \LaTeX{} и получил уверенные практические навыки работы с системой контроля версий Git.

    \begin{itemize}
        \item Использование \LaTeX{} и готового шаблона позволило мне сосредоточиться на содержании отчета, не отвлекаясь на ручное форматирование, и получить на выходе документ профессионального качества.
        \item Интерактивный тренажер Learn Git Branching помог наглядно понять сложные концепции ветвления, которые являются ядром Git.
        \item Руководство Git Immersion закрепило полученные знания через практику в командной строке, имитирующую реальный рабочий процесс.
    \end{itemize}

    Полученные навыки являются фундаментальными для любого современного IT-специалиста.
    Я планирую и дальше использовать \LaTeX{} для документирования проектов и Git для контроля версий во всех своих учебных и личных работах.

    Весь исходный код, включая программный код и текст работы
    в форме до рендеринга, доступен на GitHub по следующей ссылке: \github.

% Генерирует главу "Библиография"
    \bibliographyChapter

% --- Приложения ---
%    \appendixChapter
%
%
%    \section{Пример файла для вставки в приложение}\label{sec:-----}
%    \label{appendix:example_code}
%    \inputminted{js}{../src/appendix_example.js}

% --- Страница с декларацией ---
    \declarationPage{}

\end{document}
% vim: fdm=syntax
