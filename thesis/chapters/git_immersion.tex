\section{Уроки 1--15: Основы локального репозитория}\label{sec:1-15}
В данном разделе рассмотрены базовые операции с локальным репозиторием:

\begin{itemize}
    \item \textbf{Setup and More Setup} — установка Git, конфигурация имени и e‑mail.
    \item \textbf{Create a Project} — инициализация нового проекта (\texttt{git init}).
    \item \textbf{Checking Status} — проверка состояния репозитория (\texttt{git status}).
    \item \textbf{Making Changes} — изменение файлов.
    \item \textbf{Staging Changes} — добавление изменений в индекс (\texttt{git add}).
    \item \textbf{Staging and Committing} — подготовка и создание коммитов.
    \item \textbf{Committing Changes} — подробности создания коммита.
    \item \textbf{Changes, not Files} — отслеживание изменений, а не файлов.
    \item \textbf{History} — просмотр истории (\texttt{git log}).
    \item \textbf{Aliases} — настройка («alias») для удобства.
    \item \textbf{Getting Old Versions} — возврат к старым версиям.
    \item \textbf{Tagging versions} — создание тегов для версий.
    \item \textbf{Undoing Local Changes (before staging)} — откат изменений до стадии индексации.
    \item \textbf{Undoing Staged Changes (before committing)} — снятие из индекса.
    \item \textbf{Undoing Committed Changes} — отмена последнего коммита.
\end{itemize}

\paragraph{Описания.} Ниже — демонстрация команд и мест для скриншотов:

\begin{minted}{shell}
# Setup
git config --global user.name "Ваше Имя"
git config --global user.email "email@example.com"

# Initialize
git init myproject

# Check status
cd myproject
git status
# … далее git add, git commit …
\end{minted}

\imageWithCaption{git-immersion/staging.png}{Проверка статуса после инициализации и первые коммиты}
\label{fig:git_immersion_staging}

\paragraph{Настройка алиасов.} В этом разделе я также настроил алиасы для удобства работы с Git.
Ниже приведен пример конфигурации алиасов, которые я использую:
\begin{minted}{shell}
# //=============\\
# || Git Aliases ||
# \\=============//

alias g='git'
alias ga='git add'
alias gc='git commit -v'
alias gcam='git commit -am'
alias gd='git diff'
alias gds='git diff --staged'
alias gf='git fetch'
alias gfp='git fetch && git pull'
alias glgg='git log --graph'
alias glgga='git log --graph --decorate --all'
alias glgm='git log --graph --max-count=10'
alias gp='git push'
alias gpom='git push origin master'
alias grmc='git rm --cached'
alias gst='git status'

\end{minted}

\clearpage


\section{Уроки 16--30: Ветвление, слияние и история}\label{sec:16-30}
Этот блок посвящен веткам и управлению ими:

\begin{itemize}
    \item \textbf{Removing Commits from a Branch} — удаление последних коммитов (пример: \texttt{git reset --hard HEAD~2}).
    \item \textbf{Remove the oops tag} — удаление тега (\texttt{git tag -d oops}).
    \item \textbf{Amending Commits} — правка последнего коммита (\texttt{git commit --amend}).
    \item \textbf{Moving Files} — перемещение/переименование в Git (\texttt{git mv}).
    \item \textbf{More Structure} — организация структуры проекта.
    \item \textbf{Creating a Branch} — создание ветки (\texttt{git branch feature1}).
    \item \textbf{Navigating Branches} — переключение между ветками (\texttt{git checkout feature1}).
    \item \textbf{Changes in Main} — внесение изменений в \texttt{main}.
    \item \textbf{Viewing Diverging Branches} — обзор разветвлений (\texttt{git log --graph --decorate --all}).
    \item \textbf{Merging} — слияние веток (\texttt{git merge feature1}).
    \item \textbf{Creating a Conflict} — искусственное создание конфликта (изменение одного и того же файла).
    \item \textbf{Resolving Conflicts} — разрешение конфликта в редакторе, \texttt{git add`, `git commit}.
\end{itemize}

\paragraph{Пример команд:}
\begin{minted}{shell}
git checkout -b feature1
# вносим изменения в файл.txt
git add файл.txt
git commit -m "Добавили в feature1"

git checkout main
echo "Конфликтная строка" >> файл.txt
git commit -am "Изменение в main"

git merge feature1
# появится конфликт…
# → разрешаем, фиксируем, снова commit
\end{minted}

\clearpage


\section{Уроки 31--53: Удалённые репозитории (опционально)}\label{sec:31-53}
Если пройдены уроки с 31–53, опишите работу с GitHub:

\begin{itemize}
    \item git remote add
    \item git push
    \item git pull
    \item git clone
\end{itemize}

\paragraph{Пример:}
\begin{minted}{shell}
# Клонирование
git clone https://github.com/hoholms/uni_thesisTemplate
cd uni_thesisTemplate

# Добавить удалённый
git remote add origin git@github.com:hoholms/uni_thesisTemplate.git

# Отправить изменения
git push -u origin main

# Забрать обновления
git pull origin main
\end{minted}

\clearpage

\section*{Заключение\label{sec:conclusion}}
\addcontentsline{toc}{section}{Заключение}
В этой главе задокументирован процесс прохождения курса Git Immersion: от базовых команд до работы с удалёнными репозиториями.
Это крепко закрепило навыки работы с Git как с \ac{VCS} и подготовило к командной работе в реальном проекте.

В качестве доказательства навыков работы с Git (помимо задач выполненных во время практики), представляю ссылку на репозиторий с исходным кодом и текстом отчета: \github.
Также, в профиле GitHub можно найти другие проекты, где я применяю Git для контроля версий.

\section*{Сертификат}

На \refFigure{certif-git.jpg} представлен сертификат об успешном прохождении курса по Git на Udemy (\href{https://www.udemy.com/certificate/UC-43d72581-dc6c-4add-90a7-48f60475e13f/}{Ссылка на сертификат об окончании курса Udemy}):

\imageWithCaption{certif-git.jpg}{\href{https://www.udemy.com/course/git-and-github-bootcamp/}{The Git \& Github Bootcamp} Udemy сертификат}
\label{fig:udemy_certif_git}

