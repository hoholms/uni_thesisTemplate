\section{Установка и первая компиляция}\label{sec:install-and-first-compile}
Здесь начинается мое знакомство с \LaTeX{}.
Для работы я установил дистрибутив MacTeX и установил плагин TeXiFy-IDEA в IDE IntelliJ IDEA\@.
Процесс установки прошел гладко с использованием утилиты \texttt{brew}.
После установки я склонировал репозиторий с шаблоном и открыл его в редакторе.

На \refFigure{first_compile} показан результат первой успешной компиляции проекта.

\imageWithCaption{latex-compile.png}{Результат первой компиляции}
\label{fig:first_compile}


\section{Структура и кастомизация шаблона}\label{sec:structure-and-customization}
Основной файл \texttt{main.tex} подключает все остальные части.
Конфигурация вынесена в \texttt{config.sty}.
Исходники для примеров кода находятся в папке \texttt{src/}, а изображения в папке \texttt{thesis/images}.

Для изменения личных данных, названия работы и другой информации достаточно отредактировать блок \texttt{\textbackslash newcommand} в преамбуле \texttt{main.tex}.
\begin{minted}{latex}
        % Пример отредактированного блока
        \newcommand{\authorNameRu}{Бугаенко Никита Игоревич}
        \newcommand{\thesisTitleRu}{Отчет по практике: Изучение \LaTeX{} и системы контроля версий Git}
\end{minted}


\section{Документирование: работа с текстом, кодом и изображениями}\label{sec:documenting-with-text-code-images}
В ходе работы над отчетом я ознакомился с основными командами для форматирования текста, создания списков, вставки таблиц, изображений и блоков кода.

\subsection{Вставка блоков кода с помощью \texttt{minted}}\label{subsec:minted-code-blocks}
Пакет \texttt{minted} оказался очень удобным для отображения исходного кода.
Он автоматически подсвечивает синтаксис для множества языков.
\begin{minted}[frame=lines,
    framesep=2mm,
    linenos,
    fontsize=\small]{javascript}
        // Пример кода
        function helloGit() {
            console.log("Hello, Git Immersion!");
        }
\end{minted}

\subsection{Вставка изображений}\label{subsec:inserting-images}
Для вставки изображений (скриншотов) используется команда \texttt{\textbackslash imageWithCaption}.
Она автоматически создает рисунок с подписью и добавляет его в список иллюстраций.
На \refFigure{git_branching_example} приведен пример.

\imageWithCaption{git-branching/basics/c5t3.png}{Пример выполнения задания в Learn Git Branching}
\label{fig:git_branching_example}


\chapterConclusionSection{latex_chapter_title}
В этой главе я рассмотрел процесс установки и настройки рабочего окружения для \LaTeX{}, изучил структуру предоставленного шаблона и освоил базовые команды для наполнения документа контентом: текстом, кодом и изображениями.
Этот опыт стал основой для документирования следующих этапов практики.
